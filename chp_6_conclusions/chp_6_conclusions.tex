\documentclass{article}
    % General document formatting
    \usepackage[margin=0.7in]{geometry}
    \usepackage[parfill]{parskip}
    \usepackage[utf8]{inputenc}
    \usepackage{graphicx}
    \usepackage[superscript,biblabel]{cite}
    \usepackage{booktabs}
    
    % Related to math
    \usepackage{amsmath,amssymb,amsfonts,amsthm}
\begin{document}
\chapter{Concluding Remarks and Outlook}
\epigraph{\textit{``Mathematical analysis is as extensive as nature itself; it defines all perceptible relations, measures times, spaces, forces, temperatures; this difficult science is formed slowly, but it preserves every principle which it has once acquired; it grows and strengthens itself incessantly in the midst of the many variations and errors of the human mind.''}}{Joseph Fourier}
In this dissertation I have presented novel developments to orthogonality constrained density functional theory (OCDFT) for the treatment of electronic core-excited states. These new features aim to craft a robust computational approach for core-excited states that can provide accurate energies, properties, detailed orbital assignments, and versatility to handle unique chemical environments. 

The initial study focuses on introducing two unique features to OCDFT. First the OCDFT eigenvalue equation is augmented such that  the targeted electronic transitions are those involving hole orbitals with the \textit{lowest} eigenvalues, thus targeting the core region of the excitation spectrum. This extention, however, only allows for the computation of a single core-excited state of a given symmetry. In order to allow for full spectral simulations, the constrained multiple hole/particle (CMHP) algorithm is introduced for calculating multiple excited state solutions. Benchmark calculations were performed on a series of 23 core excitations involving core first-row elements and 17 from second-row elements. At the B3LYP/def2-QZVP level of theory, OCDFT consistently outperformed time-dependent density functional theory (TDDFT). Comparing the first-row excitations to gas-phase experimental values, OCDFT produces a mean absolute error (MAE) of 0.4 eV, while TDDFT performs significantly worse with a MAE of 11.6 eV. For the second-row excitations OCDFT and TDDFT produce MAEs of 1.6 and 31.6 eV respectively. The two theories were formally compared using a model system of two electrons in two orbitals, in order to investigate the source of error. Using this analysis, the error can be related to nonlocal Coulomb repulsion integrals that arise in the expressions of TDDFT but not in OCDFT, providing an incorrect description of the hole/particle physics of core excitations. In the case of nucleobase molecules thymine and adenine, OCDFT was also shown to produce full NEXAFS spectral simulations that provide a good comparion to experimental results.  Experimental peak positions calculated with OCDFT had an average error of  0.3 eV for thymine and 0.1 eV for adenine, with the calculated oscillator strengths providing an accurate representaton of experimental peak intensities.

The explicit treatment of relativistic effects for core-excited states was explored next through the implementation of the spin-free exact-two-component (X2C) relativistic Hamiltonian. This scalar relativistic treatment was coupled with OCDFT in order to investigate the impact of relativistic effects on core excitation energies. Our results show that for first-row core excitations X2C-OCDFT exhibits a slight improvement of 0.2 eV in the MAE compared to a non-relativistic treatment, confirming that scalar relativistic effects are not a large source of error for first-row core excitations. However, for second-row core excitations X2C-OCDFT shows a 10 eV improvement in the MAE compared to a non-relativistic treatment. Full treatment of relativistic effects via the X2C Hamiltonian made it possible to properly treat core excitations of transition metal complexes. This was shown by simulating the Ti K-edge of three Ti containing complexes TiCl$_4$, TiCpCl$_3$, and TiCp$_2$Cl$_2$ (where ``Cp'' represents cyclopentadiene). The Ti 1s core excitation energies were calculated to within approximately 5.0 eV of the experimental spectrum. This represents an improvement of 30 eV over a nonrelativistic OCDFT treatment and a 70 eV improvement over a relativistic TDDFT treatment. The atomic contributions to the pre-edge features of the Ti K-edge were critically analyzed using an atomic decomposition of the OCDFT transition dipole moment (TDM) along with a population analysis of the natural atomic orbitals (NAO). The largest atomic contributions to the total dipole are shown to be Ti s $\rightarrow$ Ti p transitions and Ti s $\rightarrow$ Cl p transitions with these two contributions accounting for roughly 80\% of the total dipole in all pre-edge transitions. Meanwhile largest change in NAO population occurs in the Ti d orbitals, with a change in Ti d population of greater than 1.0 for all pre-edge transitions. This is consistent with an interpretation that the pre-edge features of these transition metal complexes are 1s $\rightarrow$ 3d transitions that are forbidden by selection rules, yet gain intensity through mixing with Ti and ligand p orbitals. 

In an effort to provide more robust assignments of core-excited states in OCDFT, a new orbital representation is presented, known as the localized intrinsic valence virtual orbitals (LIVVOs). In this approach, the LIVVOs are derived and classified based on the intrinsic atomic orbitals (IAOs). The LIVVOs are then used to give a detailed classification of the OCDFT particle orbitals by quantifying the localized contributions. This method of classification was tested on ethanethiol and benzenethiol molecules. In both molecules, the first two excited states correspond to particle orbitals that span both thiol bond (S-H) and the adjacent S-C bond. It is difficult from visual inspection of the particle orbitals to discern the contribution of the thiol bond. Using our analysis we were able to show that the first state is dominated by the $\sigma^*_{S-H}$ LIVVO with a 58\% contribution in ethanethiol and a 61\% contribution in benzenethiol. Meanwhile the second state was more localized along the $\sigma^*_{S-C}$ LIVVO with a 57\% and 53\% contribution from ethanethiol and benzenethiol respectively. Further tests on the water monomer and dimer showed the LIVVO analysis useful in quantifying differences in the excited states due to changes in the molecular environment. The particle orbitals for the water monomer showed equal contributions from both $\sigma^*_{O-H}$ LIVVOs. When simulating the 1s core excitation spectrum of the accepting O atom in the water dimer, both $\sigma^*_{O-H}$ LIVVOs still show equal contribution suggesting that the spectrum is uneffected by accepting a hydrogen bond. However, in the spectrum of the 1s spectrum of the donating O atom, 52\% of the electron density in the particle orbital of the first core-excited is localized on the LIVVO with only 16\% of the electron density lying along the OH bond participating in the hydrogen bond. Our LIVVO analysis was able to highlight and quantify the significant effect that donating a hydrogen bond has on the local O core excitation spectrum.

Lastly I presented an extension to the CMHP algorithm in OCDFT that allows for the targeting of specific hole orbitals based on their occupation in an atomic orbital subspace. This maximum subspace occupation (MSO) method allows for targeting of core orbitals in situations where they are not the lowest energy eigenvalues in the spectrum. As a case study, the C and N K-edges of a pyrazine (C$_4$H$_4$N$_2$) molecule chemisorbed to a Si(100) surface was studied. MSO-OCDFT was able to target the organic core orbitals without first calculating core excitations from the lower energy Si 1s orbitals. This alleviates a computational difficulty that would have made this calculation infeasible in the original CMHP algorithm. Subsequent LIVVO analysis of the pyrazine spectrum reveals interesting changes upon chemisorption to the surface. For example the first state in the C 1s spectrum maintains similar localization about the $\pi^*$ LIVVO of the C=C double bond to the gas phase spectrum. In the gas phase the particle orbital has 78.2\%$\pi^*_{C-C}$ LIVVO, when chemisorbed, it has 73.4\% overlap with this LIVVO. As a result the intensity is identical in both cases ($f_{abs} = 0.02$). Meanwhile the localization of all states in the $\sigma^*$ manifold drop by 35\% in all cases. Leading to a dramatic difference in intensity between the $\pi^*$ and $\sigma^*$ manifolds in the chemisorbed spectrum.

This dissertation establishes OCDFT as an accurate and cost effective method for the computation of core-excited states. The theory has been significantly enhanced with new features that allow for the treatment of a wide range of different chemical systems including chemisorbed organic molecules and transition metal complexes. Further work on the theory is required in order to encompass a more complete treatment of the entire field of X-ray spectroscopy. While the work presented here was limited to K-edge applications, extensions to treat L-edge spectroscopy are possible. This would require pairing OCDFT with a higher-order relativistic treatment to account for the spin-orbit splitting that becoms a factor in 2p core-excitations. Another desirable extension would be a method to treat vibrational effects within the core spectrum, resulting in a more realistic representation of the spectral resonances. One can imagine coupling OCDFT with a discrete variable representation (DVR) approach for the treatment of vibrations. Lastly, OCDFT could be extended for the treatment of X-ray emission spectroscopy. This problem would require the calculation of an approximate intermediate state in which a core hole is introduced. This core hole would then be the LUMO for all subsequent excited state calculations and thus the lowest energy solutions to the OCDFT eigenvalue equation would be the x-ray emission energies.
\end{document}